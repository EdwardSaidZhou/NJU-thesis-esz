%% 使用 njuthesis 文档类生成南京大学学位论文的示例文档
%%
%% 作者:胡海星,starfish (at) gmail (dot) com
%% 项目主页: http://haixing-hu.github.io/nju-thesis/
%%
%% 本样例文档中用到了吕琦同学的博士论文的提高和部分内容,在此对他表示感谢。
%%
\documentclass[phd]{njuthesis}
%% njuthesis 文档类的可选参数有:
%%   nobackinfo 取消封二页导师签名信息。注意,按照南大的规定,是需要签名页的。
%%   phd/master/bachelor 选择博士/硕士/学士论文

\setCJKmainfont{FZLTSK.TTF}
\setCJKmonofont{FZLTSK.TTF}
\setCJKsansfont{FZLTSK.TTF}
%\setmainfont{Times New Roman}
%%%%%%%%%%%%% 编译模式:Texwork下 XeLaTeX+MakeIndex+BibTex %%%%%%%

\usepackage{chngcntr}
\counterwithout{figure}{chapter}
\counterwithout{table}{chapter}

% 使用 blindtext 宏包自动生成章节文字
% 这仅仅是用于生成样例文档,正式论文中一般用不到该宏包
\usepackage[math]{blindtext}
\usepackage{ctex}
\usepackage{xcolor}
\urlstyle{same}
\usepackage{lipsum}
\usepackage[backend=biber,style=gb7714-2015,defernumbers=true,sortlocale=zh__pinyin,sorting=gb7714-2015]{biblatex}
\addbibresource{articleandbook.bib}
\addbibresource{online.bib}
\addbibresource{bib3.bib}
\usepackage{scrextend}
\deffootnote{1.5em}{1em}{%
  \makebox[1.5em][l]{\thefootnotemark}}
\defbibcheck{english}{\iffieldequalstr{langid}{english}{}{\skipentry}}
\defbibcheck{chinese}{\iffieldequalstr{langid}{中文;}{}{\skipentry}}
\usepackage{makeidx}
\makeindex
\newcommand*{\main}[1]{\textbf{\hyperpage{#1}}}  
%%%%%%%%%%%%%%%%%%%%%%%%%%%%%%%%%%%%%%%%%%%%%%%%%%%%%%%%%%%%%%%%%%%%%%%%%%%%%%%
% 设置《国家图书馆封面》的内容,仅博士论文才需要填写

% 设置论文按照《中国图书资料分类法》的分类编号
\classification{0175.2}
% 论文的密级。需按照GB/T 7156-2003标准进行设置。预定义的值包括:
% - \openlevel,表示公开级:此级别的文献可在国内外发行和交换。
% - \controllevel,表示限制级:此级别的文献内容不涉及国家秘密,但在一定时间内
%   限制其交流和使用范围。
% - \confidentiallevel,表示秘密级:此级别的文献内容涉及一般国家秘密。
% - \clasifiedlevel,表示机密级:此级别的文献内容涉及重要的国家秘密 。
% - \mostconfidentiallevel,表示绝密级:此级别的文献内容涉及最重要的国家秘密。
% 此属性可选,默认为\openlevel,即公开级。
\securitylevel{\controllevel}
% 设置论文按照《国际十进分类法UDC》的分类编号
% 该编号可在下述网址查询:http://www.udcc.org/udcsummary/php/index.php?lang=chi
\udc{004.72}
% 国家图书馆封面上的论文标题第一行,不可换行。此属性可选,默认值为通过\title设置的标题。
\nlctitlea{学科}
% 国家图书馆封面上的论文标题第二行,不可换行。此属性可选,默认值为空白。
\nlctitleb{理论研究}
% 国家图书馆封面上的论文标题第三行,不可换行。此属性可选,默认值为空白。
\nlctitlec{}
% 导师的单位名称及地址
\supervisorinfo{南京大学国际关系研究院~~南京市仙林大道163号~~210023}
% 答辩委员会主席
\chairman{某某~~教授}
% 第一位评阅人
\reviewera{~~教授}
% 第二位评阅人
\reviewerb{~~教授}
% 第三位评阅人
\reviewerc{~~教授}
% 第四位评阅人
\reviewerd{~~研究员}

%%%%%%%%%%%%%%%%%%%%%%%%%%%%%%%%%%%%%%%%%%%%%%%%%%%%%%%%%%%%%%%%%%%%%%%%%%%%%%%
% 设置论文的中文封面

% 论文标题,不可换行
\titlea{学科}
\titleb{理论研究}
% 如果论文标题过长,可以分两行,第一行用\titlea{}定义,第二行用\titleb{}定义
% 并将上面的标题命令设为\title{}以置空标题
% \titlea{半轻衰变$D^+\to \omega(\phi)e^+\nu_e$的研究}
% \titleb{和弱衰变$J/\psi \to D_s^{(*)-}e^+\nu_e$的寻找}

% 论文作者姓名
\author{张三}
% 论文作者联系电话
\telphone{188XXXXXXXX}
% 论文作者电子邮件地址
\email{zhangsan@nju.edu.cn}
% 论文作者学生证号
\studentnum{DGXXXXXXX}
% 论文作者入学年份(年级)
\grade{2020}
% 导师姓名职称
\supervisor{某某~~教授}
% 导师的联系电话
\supervisortelphone{150XXXXXXXX}
% 论文作者的学科与专业方向
\major{学科}
% 论文作者的研究方向
\researchfield{某某研究}
% 论文作者所在院系的中文名称
\department{某某研究院}
% 论文作者所在学校或机构的名称。此属性可选,默认值为``南京大学''。
\institute{南京大学}
% 论文的提交日期,需设置年、月、日。
\submitdate{2022年5月10日}
% 论文的答辩日期,需设置年、月、日。
\defenddate{2022年6月1日}
% 论文的定稿日期,需设置年、月、日。此属性可选,默认值为最后一次编译时的日期,精确到日。
%% \date{2013年5月1日}

%%%%%%%%%%%%%%%%%%%%%%%%%%%%%%%%%%%%%%%%%%%%%%%%%%%%%%%%%%%%%%%%%%%%%%%%%%%%%%%
% 设置论文的英文封面

% 论文的英文标题,不可换行
\englishtitle{Field}
% 论文作者姓名的拼音
\englishauthor{ZHANG San}
% 导师姓名职称的英文
\englishsupervisor{Professor Anyone}
% 论文作者学科与专业的英文名
\englishmajor{Field}
% 论文作者所在院系的英文名称
\englishdepartment{XX College}
% 论文作者所在学校或机构的英文名称。此属性可选,默认值为``Nanjing University''。
\englishinstitute{Nanjing University}
% 论文完成日期的英文形式,它将出现在英文封面下方。需设置年、月、日。日期格式使用美国的日期
% 格式,即``Month day, year'',其中``Month''为月份的英文名全称,首字母大写;``day''为
% 该月中日期的阿拉伯数字表示;``year''为年份的四位阿拉伯数字表示。此属性可选,默认值为最后
% 一次编译时的日期。
\englishdate{May 1, 2022}

%%%%%%%%%%%%%%%%%%%%%%%%%%%%%%%%%%%%%%%%%%%%%%%%%%%%%%%%%%%%%%%%%%%%%%%%%%%%%%%
% 设置论文的中文摘要

% 设置中文摘要页面的论文标题及副标题的第一行。
% 此属性可选,其默认值为使用|\title|命令所设置的论文标题
 \abstracttitlea{学科}
% 设置中文摘要页面的论文标题及副标题的第二行。
% 此属性可选,其默认值为空白
 \abstracttitleb{理论研究}

%%%%%%%%%%%%%%%%%%%%%%%%%%%%%%%%%%%%%%%%%%%%%%%%%%%%%%%%%%%%%%%%%%%%%%%%%%%%%%%
% 设置论文的英文摘要

% 设置英文摘要页面的论文标题及副标题的第一行。
% 此属性可选,其默认值为使用|\englishtitle|命令所设置的论文标题
\englishabstracttitlea{Field}
% 设置英文摘要页面的论文标题及副标题的第二行。
% 此属性可选,其默认值为空白
\englishabstracttitleb{Theory Research}


%%%%%%%%%%%%%%%%%%%%%%%%%%%%%%%%%%%%%%%%%%%%%%%%%%%%%%%%%%%%%%%%%%%%%%%%%%%%%%%
\begin{document}

%%%%%%%%%%%%%%%%%%%%%%%%%%%%%%%%%%%%%%%%%%%%%%%%%%%%%%%%%%%%%%%%%%%%%%%%%%%%%%%

% 制作国家图书馆封面(博士学位论文才需要)
\makenlctitle
% 制作中文封面
\maketitle
% 制作英文封面
\makeenglishtitle


%%%%%%%%%%%%%%%%%%%%%%%%%%%%%%%%%%%%%%%%%%%%%%%%%%%%%%%%%%%%%%%%%%%%%%%%%%%%%%%
% 开始前言部分
\frontmatter

%%%%%%%%%%%%%%%%%%%%%%%%%%%%%%%%%%%%%%%%%%%%%%%%%%%%%%%%%%%%%%%%%%%%%%%%%%%%%%%
% 论文的中文摘要
\begin{abstract}
复杂网络的研究可上溯到20世纪60年代对ER网络的研究。90年后代随着Internet
的发展,以及对人类社会、通信网络、生物网络、社交网络等各领域研究的深入,
发现了小世界网络和无尺度现象等普适现象与方法。对复杂网络的定性定量的科
学理解和分析,已成为如今网络时代科学研究的一个重点课题。

在此背景下,由于云计算时代的到来,本文针对面向云计算的数据中心网络基础
设施设计中的若干问题,进行了几方面的研究。………………
% 中文关键词。关键词之间用中文全角分号隔开,末尾无标点符号。
\keywords{小世界理论;网络模型;数据中心}
\end{abstract}

%%%%%%%%%%%%%%%%%%%%%%%%%%%%%%%%%%%%%%%%%%%%%%%%%%%%%%%%%%%%%%%%%%%%%%%%%%%%%%%
% 论文的英文摘要
\begin{englishabstract}
\blindtext
% 英文关键词。关键词之间用英文半角逗号隔开,末尾无符号。
\englishkeywords{Small World; Network Model; Data Center}
\end{englishabstract}

%%%%%%%%%%%%%%%%%%%%%%%%%%%%%%%%%%%%%%%%%%%%%%%%%%%%%%%%%%%%%%%%%%%%%%%%%%%%%%%
% 论文的前言,应放在目录之前,中英文摘要之后
%
\begin{preface}

复杂网络的研究可上溯到20世纪60年代对ER网络的研究。90年后代随着Internet
的发展,以及对人类社会、通信网络、生物网络、社交网络等各领域研究的深入,
发现了小世界网络和无尺度现象等普适现象与方法。对复杂网络的定性定量的科
学理解和分析,已成为如今网络时代科学研究的一个重点课题。

在此背景下,由于云计算时代的到来,本文针对面向云计算的数据中心网络基础
设施设计中的若干问题,进行了几方面的研究。本文的创造性研究成果主要如下
几方面:

\begin{enumerate}
\item 基于簇划分的思想,提出并设计了WarpNet网络模型。该网络模型基于随机
  散列,以节点微路由链接多种散列分布,实现网络互联。并对网络的带宽等指
  标进行理论分析并给出定量描述。最后对比了理论分析、仿真测试结果,并在
  实际物理环境中进系真实部署,通过6节点的小规模实验以及1000节点虚拟机的
  大规模实验,表明该模型的理论分析、仿真测试与实际实验吻合,并在网络性
  能、容错能力、伸缩性灵活性方面得到较大提升。
\item 提出DS小世界模型并构造SIDN网络,解决了把小世界理论应用于数据中心
  网络布局构建中的最大度限制问题。分析了在带有最大度限制约束下,所构成
  网络的平均直径、网络总带宽、对故障的容错能力等各项网络参数。理论分析
  与仿真实验证明,SIDN网络具有很好的扩展能力,网络总带宽与网络规模成近
  似线性增长的关系;具有很强的容错能力,链路损坏与节点损坏几乎无法破坏
  网络的联通性,故障率对网络性能的影响与破坏节点/链路占总资源比率线性相
  关。
\item 分析了无尺度网络在数据中心网络构建应用中的理论方面问题。在引入节
  点最大度限制之后,给出无尺度网络的各项网络参数。并进一步分析了交换机
  节点以及计算节点两种角色在不同比率的组合下对网络性能的影响,给出最高
  性价比的比率参数。最后通过理论分析与仿真实验证明,在引入了无尺度现象
  之后,提高了网络的聚类系数,从而显著的提升了网络的性能。

\item 针对网络模型研究这一类工作的共性,设计构造通用验证平台系统。以海
  量虚拟机和虚拟分布式交换机的形式,实现了基于少量物理节点,对大规模节
  点的模拟。其模拟运行的过程与真实运行在实现层面完全一致,运行的结果与
  真实环境线性相关。除为本文所涉若干网络模型提供验证外,可进一步推广到
  更为广泛的领域,为各种网络模型及路由算法的研究工作,提供分析、指导与
  验证。
\end{enumerate}

复杂网络的研究可上溯到20世纪60年代对ER网络的研究。90年后代随着Internet
的发展,以及对人类社会、通信网络、生物网络、社交网络等各领域研究的深入,
发现了小世界网络和无尺度现象等普适现象与方法。对复杂网络的定性定量的科
学理解和分析,已成为如今网络时代科学研究的一个重点课题。

在此背景下,由于云计算时代的到来,本文针对面向云计算的数据中心网络基础
设施设计中的若干问题,进行了几方面的研究。本文的创造性研究成果主要如下
几方面:

\begin{enumerate}
\item 基于簇划分的思想,提出并设计了WarpNet网络模型。该网络模型基于随机
  散列,以节点微路由链接多种散列分布,实现网络互联。并对网络的带宽等指
  标进行理论分析并给出定量描述。最后对比了理论分析、仿真测试结果,并在
  实际物理环境中进系真实部署,通过6节点的小规模实验以及1000节点虚拟机的
  大规模实验,表明该模型的理论分析、仿真测试与实际实验吻合,并在网络性
  能、容错能力、伸缩性灵活性方面得到较大提升。
\item 提出DS小世界模型并构造SIDN网络,解决了把小世界理论应用于数据中心
  网络布局构建中的最大度限制问题。分析了在带有最大度限制约束下,所构成
  网络的平均直径、网络总带宽、对故障的容错能力等各项网络参数。理论分析
  与仿真实验证明,SIDN网络具有很好的扩展能力,网络总带宽与网络规模成近
  似线性增长的关系;具有很强的容错能力,链路损坏与节点损坏几乎无法破坏
  网络的联通性,故障率对网络性能的影响与破坏节点/链路占总资源比率线性相
  关。
\item 分析了无尺度网络在数据中心网络构建应用中的理论方面问题。在引入节
  点最大度限制之后,给出无尺度网络的各项网络参数。并进一步分析了交换机
  节点以及计算节点两种角色在不同比率的组合下对网络性能的影响,给出最高
  性价比的比率参数。最后通过理论分析与仿真实验证明,在引入了无尺度现象
  之后,提高了网络的聚类系数,从而显著的提升了网络的性能。

\item 针对网络模型研究这一类工作的共性,设计构造通用验证平台系统。以海
  量虚拟机和虚拟分布式交换机的形式,实现了基于少量物理节点,对大规模节
  点的模拟。其模拟运行的过程与真实运行在实现层面完全一致,运行的结果与
  真实环境线性相关。除为本文所涉若干网络模型提供验证外,可进一步推广到
  更为广泛的领域,为各种网络模型及路由算法的研究工作,提供分析、指导与
  验证。
\end{enumerate}

\vspace{1cm}
\begin{flushright}
张三\\
2022年夏于南京大学
\end{flushright}

\end{preface}

%%%%%%%%%%%%%%%%%%%%%%%%%%%%%%%%%%%%%%%%%%%%%%%%%%%%%%%%%%%%%%%%%%%%%%%%%%%%%%%
% 生成论文目次
\tableofcontents

%%%%%%%%%%%%%%%%%%%%%%%%%%%%%%%%%%%%%%%%%%%%%%%%%%%%%%%%%%%%%%%%%%%%%%%%%%%%%%%
% 生成插图清单。如无需插图清单则可注释掉下述语句。
\listoffigures

%%%%%%%%%%%%%%%%%%%%%%%%%%%%%%%%%%%%%%%%%%%%%%%%%%%%%%%%%%%%%%%%%%%%%%%%%%%%%%%
% 生成附表清单。如无需附表清单则可注释掉下述语句。
\listoftables

%%%%%%%%%%%%%%%%%%%%%%%%%%%%%%%%%%%%%%%%%%%%%%%%%%%%%%%%%%%%%%%%%%%%%%%%%%%%%%%
% 开始正文部分
\mainmatter

%%%%%%%%%%%%%%%%%%%%%%%%%%%%%%%%%%%%%%%%%%%%%%%%%%%%%%%%%%%%%%%%%%%%%%%%%%%%%%%
% 学位论文的正文应以《绪论》作为第一章
\chapter*{绪论}\label{chapter_introduction}
\markboth{绪论}{} 
\phantomsection 
\addcontentsline{toc}{chapter}{绪论} 


\section{问题的提出}

\subsection{选题背景}

习近平总书记指出,“当前中国处于近代以来最好的发展时期,世界处于百年未有之大变局,两者同步交织、相互激荡”。
\chapter{随机网络模型}\label{chapter1}
\section{随机网络背景与研究现状}
\Blindtext
\begin{table}
    \centering
    \caption{\label{ingo}主要政府间国际性及亚洲区域性国际组织总部所在地}
    \begin{tabular}{p{11cm}<{\raggedright}p{2cm}<{\raggedright}}
    \toprule
    \textbf{国际组织}                                                                                     & \textbf{所在地} \\ \midrule
    联合国教科文组织,经合组织,国际能源署                                                                               & 巴黎           \\
    国际清算银行                                                                                            & 巴塞尔          \\
    \textit{亚洲基础设施投资银行,国际竹藤组织}                                                                        & \textit{北京}  \\
    万国邮政联盟                                                                                            & 伯尔尼          \\
    世界海关组织,北极条约组织,欧洲联盟,北大西洋公约组织                                                                       & 布鲁塞尔         \\
    联合国大学                                                                                             & 东京           \\
    国际海洋考察理事会                                                                                         & 哥本哈根         \\
    世界自然基金会                                                                                           & 格朗           \\
    国际法庭,国际刑事法庭,禁止化学武器组织,常设仲裁庭                                                                        & 海牙           \\
    多边投资担保机构,国际货币基金会,世界银行集团,国际投资争端解决中心,国际开发协会,国际金融公司                                                  & 华盛顿          \\
    南亚区域合作联盟                                                                                          & 加德满都         \\
    国际刑事警察组织                                                                                          & 里昂           \\
    国际海事组织,英联邦                                                                                        & 伦敦           \\
    联合国粮农署,国际农业发展基金,世界粮食计划署                                                                           & 罗马           \\
    国际奥委会                                                                                             & 洛桑           \\
    世界旅游组织                                                                                            & 马德里          \\
    亚洲开发银行                                                                                            & 马尼拉          \\
    国际民用航空组织                                                                                          & 蒙特利尔         \\
    联合国环境署                                                                                            & 内罗毕          \\
    联合国(联合国大会,联合国安全理事会,联合国儿童基金会,联合国开发计划署,联合国计划署等),国际救援委员会                                             & 纽约           \\
    各国议会联盟,国际标准组织,国际劳工组织,国际电信联盟,世界卫生组织,世界知识产权组织,世界气象组织,世界贸易组织,联合国人权高专办,联合国难民高专办,红十字会与红新月会国际联合会,国际移民组织 & 日内瓦          \\
    \textit{新开发银行}                                                                                    & \textit{上海}  \\
    世界足联                                                                                              & 苏黎世          \\
    石油输出国组织,国际原子能组织,联合国工业发展组织                                                                         & 维也纳          \\
    东南亚国家联盟                                                                                           & 雅加达          \\ \bottomrule
    \end{tabular}
\end{table}
\section{WarpNet网络模型构建}\label{sec:warpnet_construction}
\subsection{网络结构}
\Blindtext
\subsection{双层拓扑结构}
\Blindtext
\section{路由算法}
\subsection{基于flood的路由发现算法}
\Blindtext
\section{网络性能分析}
\blindtext
\subsection{连通性与互联通率}
\Blindtext
\subsection{路由跳数}
\Blindtext
\subsection{总带宽}
\Blindtext
\subsection{故障对网络的影响}
\Blindtext
\section{仿真分析}
\Blindtext
\section{实验验证}
\Blindtext
\section{小结}
\blindtext
%\include{chapters/chapter2}
%\include{chapters/chapter3}
%\include{chapters/chapter4}
%\include{chapters/chapter5}
%\include{chapters/chapter6}
\chapter*{结语} 
\markboth{结语}{}  
\phantomsection 
\addcontentsline{toc}{chapter}{结语} 

在给出了在引入节点最大度限制之后,利用分治和递归的思想,对无尺度网络
进行多层构建,对所构造的网络进行度-度相关性,以及聚类性分析。

\begin{table}
  \centering
  \begin{tabular}{cccp{38mm}}
    \toprule
    \textbf{文档域类型} & \textbf{Java类型} & \textbf{宽度(字节)} & \textbf{说明} \\
    \midrule
    BOOLEAN  & boolean &  1  & \\
    CHAR     & char    &  2  & UTF-16字符 \\
    BYTE     & byte    &  1  & 有符号8位整数 \\
    SHORT    & short   &  2  & 有符号16位整数 \\
    INT      & int     &  4  & 有符号32位整数 \\
    LONG     & long    &  8  & 有符号64位整数 \\
    STRING   & String  &  字符串长度  & 以UTF-8编码存储 \\
    DATE     & java.util.Date & 8 & 距离GMT时间1970年1月1日0点0分0秒的毫秒数 \\
    BYTE\_ARRAY & byte$[]$ & 数组长度 & 用于存储二进制值 \\
    BIG\_INTEGER & java.math.BigInteger & 和具体值有关 & 任意精度的长整数 \\
    BIG\_DECIMAL & java.math.BigDecimal & 和具体值有关 & 任意精度的十进制实数 \\
    \bottomrule
  \end{tabular}
  \caption{测试表格}\label{table:test5}
\end{table}

用于测试表格。随后分析了无尺度网络构造过程中,交换机节点与数
据节点的角色区别,分析了两者在不同比率下形成的网络形态,以及对网络性能造成的影响。

通过理论分析和仿真实验,分析并找出比率因子q的最佳取值。此外,无尺度现象
的引入提高了网络的聚类系数,从而在不失灵活性可靠性的基础上,进一步提升
了网络的性能。

将关注点转移到交换机\index{交换机}本身。由于图论难以描述数据中心
网络中的交换设备,因此放弃基于图的抽象模型,转而基于多维簇划分的思想,提出并设计
了WarpNet网络模型。

该网络模型突破了基于图描述的局限性,并对网络的带宽等指标进行理论分析并
给出定量描述。最后对比了理论分析、仿真测试结果,并在实际物理环境中进系
真实部署,通过6节点的小规模实验以及1000节点虚拟机的大规模实验,表明该模
型的理论分析、仿真测试与实际实验吻合,并在网络性能、容错能力、伸缩性灵
活性方面得到了进一步的提升。

在第\ref{chapter_introduction}章中,针对网络模型研究这一类工作的共性,设计构造通
用验证平台系统。以海量虚拟机和虚拟分布式交换机的形式,实现了基于少量物理节点,对
大规模节点的模拟。其模拟运行的过程与真实运行在实现层面完全一致,运行的结果与真实
环境线性相关。除为本文所涉若干网络模型提供验证外,可进一步推广到更为广泛的领域,
为各种网络模型及路由算法的研究工作,提供分析、指导与验证。

%%%%%%%%%%%%%%%%%%%%%%%%%%%%%%%%%%%%%%%%%%%%%%%%%%%%%%%%%%%%%%%%%%%%%%%%%%%%%%%
\cleardoublepage
\phantomsection
\addcontentsline{toc}{chapter}{参考文献}
%\nocite{*}
\printbibheading
\defbibfilter{books}{type=book or type=incollection}
\printbibliography[check=chinese,filter=books,heading=subbibliography,title={中文书籍},resetnumbers=true]
\printbibliography[check=chinese,type=article,heading=subbibliography,title={中文论文},resetnumbers=true]
\printbibliography[check=chinese,type=online,heading=subbibliography,title={中文网络资源},resetnumbers=true]
\printbibliography[check=chinese,nottype=book,nottype=incollection,nottype=article,nottype=online,heading=subbibliography,title={中文其他资源},resetnumbers=true]
\printbibliography[check=english,filter=books,heading=subbibliography,title={英文书籍},resetnumbers=true]
\printbibliography[check=english,type=article,heading=subbibliography,title={英文论文},resetnumbers=true]
\printbibliography[check=english,type=online,heading=subbibliography,title={英文网络资源},resetnumbers=true]
\printbibliography[check=english,nottype=book,nottype=incollection,nottype=article,nottype=online,heading=subbibliography,title={英文其他资源},resetnumbers=true]

%档案使用"manual",Zotero中为“报告”,导出的bib修改institution为organization

%%%%%%%%%%%%%%%%%%%%%%%%%%%%%%%%%%%%%%%%%%%%%%%%%%%%%%%%%%%%%%%%%%%%%%%%%%%%%%%
% 致谢,应放在《结论》之后
\begin{acknowledgement}
  首先感谢我的母亲韦春花对我的支持。其次感谢我的导师陈近南对我的精心指导和热心帮助。接下来,
  感谢我的师兄茅十八和风际中,他们阅读了我的论文草稿并提出了很有价值的修改建议。

  最后,感谢我亲爱的老婆们:双儿、苏荃、阿珂、沐剑屏、曾柔、建宁公主、方怡,感谢
  你们在生活上对我无微不至的关怀和照顾。我爱你们!
\end{acknowledgement}

%%%%%%%%%%%%%%%%%%%%%%%%%%%%%%%%%%%%%%%%%%%%%%%%%%%%%%%%%%%%%%%%%%%%%%%%%%%%%%%
% 书籍附件
\backmatter
%%%%%%%%%%%%%%%%%%%%%%%%%%%%%%%%%%%%%%%%%%%%%%%%%%%%%%%%%%%%%%%%%%%%%%%%%%%%%%%
% 作者简历与科研成果页,应放在backmatter之后
\begin{resume}
  \begin{firstauthor} 
  \item 《XXX》,载《XXX》(CSSCI),2021年第1期。
  \end{firstauthor}
  \begin{secondauthor}
  \item 《XXXX》,载《XXXX》, 2021年第1期。
  \end{secondauthor}
  \end{resume}

%%%%%%%%%%%%%%%%%%%%%%%%%%%%%%%%%%%%%%%%%%%%%%%%%%%%%%%%%%%%%%%%%%%%%%%%%%%%%%%
% 生成《学位论文出版授权书》页面,应放在最后一页
\makelicense

%%%%%%%%%%%%%%%%%%%%%%%%%%%%%%%%%%%%%%%%%%%%%%%%%%%%%%%%%%%%%%%%%%%%%%%%%%%%%%%
\end{document}
