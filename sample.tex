%% 使用 njuthesis 文档类生成南京大学学位论文的示例文档
%%
%% 作者:胡海星,starfish (at) gmail (dot) com
%% 项目主页: http://haixing-hu.github.io/nju-thesis/
%%
%% 本样例文档中用到了吕琦同学的博士论文的提高和部分内容,在此对他表示感谢。
%%
\documentclass[phd]{njuthesis}
%% njuthesis 文档类的可选参数有:
%%   nobackinfo 取消封二页导师签名信息。注意,按照南大的规定,是需要签名页的。
%%   phd/master/bachelor 选择博士/硕士/学士论文

% 使用 blindtext 宏包自动生成章节文字
% 这仅仅是用于生成样例文档,正式论文中一般用不到该宏包
\usepackage[math]{blindtext}


\setCJKmainfont{FZXSSJW.TTF}
\setCJKmonofont{FZXSSJW.TTF}
\setCJKsansfont{SimHei}
%%%%%%%%%%%%% 编译模式:Texwork下 XeLaTeX+MakeIndex+BibTex %%%%%%%
\usepackage{bookmark}
\usepackage{chngcntr}
%\counterwithout{figure}{chapter}
%\counterwithout{table}{chapter}

\setlength{\parskip}{0ex}

% 使用 blindtext 宏包自动生成章节文字
% 这仅仅是用于生成样例文档,正式论文中一般用不到该宏包
\usepackage[math]{blindtext}
\usepackage{anyfontsize}
\usepackage{ctex}
\usepackage{xcolor}
\usepackage{setspace}
\urlstyle{same}
\usepackage{lipsum}
\usepackage[backend=biber,style=gb7714-2015,defernumbers=true,maxnames=3,gbbiblabel=bracket,gbfootbib=true,gbfieldtype=true,gblocal=gb7714-2015,gbnamefmt=lowercase,gbnamefmt=givenahead,giveninits=false,doi=false,isbn=false,sorting=multipinyin,sortlocale=zh__gb2312han]{biblatex}
\AtEveryCitekey{\clearfield{urldate}}
\AtEveryBibitem{\clearfield{urldate}}
\addbibresource{bib.bib}
%档案使用"manual",Zotero中为“报告”,导出的bib修改institution为organization
\usepackage{scrextend}
\deffootnote{1.5em}{1em}{%
  \makebox[1.5em][l]{\thefootnotemark}}
\defbibcheck{english}{\iffieldequalstr{langid}{english}{}{\skipentry}}
\defbibcheck{chinese}{\iffieldequalstr{langid}{chinese}{}{\skipentry}}
\usepackage{makeidx}
\makeindex
\newcommand*{\main}[1]{\textbf{\hyperpage{#1}}} 
\renewcommand\multicitedelim{\addsemicolon\space}

\usepackage[nottoc]{tocbibind}

\makeatletter
\renewcommand{\l@section}{\@dottedtocline{2}{2em}{2em}}
\renewcommand{\l@subsection}{\@dottedtocline{2}{4em}{2.75em}}
\makeatother

\DeclareSortingTemplate{multipinyin}{
       \sort{
         \field{presort}
       }
       \sort{
         \field{lansortorder}%language
       }
       \sort{
         \field{sortkey}
       }
       \sort{%[direction=descending]
         \field{sortname}
         \field{author}
         \field{editor}
         \field{translator}
      }
      \sort{
        \field{sortyear}
        \field{year}
      }
      \sort{
        \field{sorttitle}
        \field{title}
      }
    }

\DeclareSortingNamekeyScheme{
  \keypart{
    \namepart{given}
  }
  \keypart{
    \namepart{prefix}
  }
  \keypart{
    \namepart{family}
  }
  \keypart{
    \namepart{suffix}
  }
}

\DeclareNameAlias{sortname}{default}
%%%%%%%%%%%%%%%%%%%%%%%%%%%%%%%%%%%%%%%%%%%%%%%%%%%%%%%%%%%%%%%%%%%%%%%%%%%%%%%
% 设置《国家图书馆封面》的内容,仅博士论文才需要填写

% 设置论文按照《中国图书资料分类法》的分类编号
\classification{0175.2}
% 论文的密级。需按照GB/T 7156-2003标准进行设置。预定义的值包括:
% - \openlevel,表示公开级:此级别的文献可在国内外发行和交换。
% - \controllevel,表示限制级:此级别的文献内容不涉及国家秘密,但在一定时间内
%   限制其交流和使用范围。
% - \confidentiallevel,表示秘密级:此级别的文献内容涉及一般国家秘密。
% - \clasifiedlevel,表示机密级:此级别的文献内容涉及重要的国家秘密 。
% - \mostconfidentiallevel,表示绝密级:此级别的文献内容涉及最重要的国家秘密。
% 此属性可选,默认为\openlevel,即公开级。
\securitylevel{\controllevel}
% 设置论文按照《国际十进分类法UDC》的分类编号
% 该编号可在下述网址查询:http://www.udcc.org/udcsummary/php/index.php?lang=chi
\udc{004.72}
% 国家图书馆封面上的论文标题第一行,不可换行。此属性可选,默认值为通过\title设置的标题。
\nlctitlea{学科}
% 国家图书馆封面上的论文标题第二行,不可换行。此属性可选,默认值为空白。
\nlctitleb{理论研究}
% 国家图书馆封面上的论文标题第三行,不可换行。此属性可选,默认值为空白。
\nlctitlec{}
% 导师的单位名称及地址
\supervisorinfo{南京大学国际关系研究院~~南京市仙林大道163号~~210023}
% 答辩委员会主席
\chairman{某某~~教授}
% 第一位评阅人
\reviewera{~~教授}
% 第二位评阅人
\reviewerb{~~教授}
% 第三位评阅人
\reviewerc{~~教授}
% 第四位评阅人
\reviewerd{~~研究员}

%%%%%%%%%%%%%%%%%%%%%%%%%%%%%%%%%%%%%%%%%%%%%%%%%%%%%%%%%%%%%%%%%%%%%%%%%%%%%%%
% 设置论文的中文封面

% 论文标题,不可换行
\titlea{学科}
\titleb{理论研究}
% 如果论文标题过长,可以分两行,第一行用\titlea{}定义,第二行用\titleb{}定义
% 并将上面的标题命令设为\title{}以置空标题
% \titlea{半轻衰变$D^+\to \omega(\phi)e^+\nu_e$的研究}
% \titleb{和弱衰变$J/\psi \to D_s^{(*)-}e^+\nu_e$的寻找}

% 论文作者姓名
\author{张三}
% 论文作者联系电话
\telphone{188XXXXXXXX}
% 论文作者电子邮件地址
\email{zhangsan@nju.edu.cn}
% 论文作者学生证号
\studentnum{DGXXXXXXX}
% 论文作者入学年份(年级)
\grade{2020}
% 导师姓名职称
\supervisor{某某~~教授}
% 导师的联系电话
\supervisortelphone{150XXXXXXXX}
% 论文作者的学科与专业方向
\major{学科}
% 论文作者的研究方向
\researchfield{某某研究}
% 论文作者所在院系的中文名称
\department{某某研究院}
% 论文作者所在学校或机构的名称。此属性可选,默认值为``南京大学''。
\institute{南京大学}
% 论文的提交日期,需设置年、月、日。
\submitdate{2022年5月10日}
% 论文的答辩日期,需设置年、月、日。
\defenddate{2022年6月1日}
% 论文的定稿日期,需设置年、月、日。此属性可选,默认值为最后一次编译时的日期,精确到日。
%% \date{2013年5月1日}

%%%%%%%%%%%%%%%%%%%%%%%%%%%%%%%%%%%%%%%%%%%%%%%%%%%%%%%%%%%%%%%%%%%%%%%%%%%%%%%
% 设置论文的英文封面

% 论文的英文标题,不可换行
\englishtitle{Field}
% 论文作者姓名的拼音
\englishauthor{ZHANG San}
% 导师姓名职称的英文
\englishsupervisor{Professor Anyone}
% 论文作者学科与专业的英文名
\englishmajor{Field}
% 论文作者所在院系的英文名称
\englishdepartment{XX College}
% 论文作者所在学校或机构的英文名称。此属性可选,默认值为``Nanjing University''。
\englishinstitute{Nanjing University}
% 论文完成日期的英文形式,它将出现在英文封面下方。需设置年、月、日。日期格式使用美国的日期
% 格式,即``Month day, year'',其中``Month''为月份的英文名全称,首字母大写;``day''为
% 该月中日期的阿拉伯数字表示;``year''为年份的四位阿拉伯数字表示。此属性可选,默认值为最后
% 一次编译时的日期。
\englishdate{May 1, 2022}

%%%%%%%%%%%%%%%%%%%%%%%%%%%%%%%%%%%%%%%%%%%%%%%%%%%%%%%%%%%%%%%%%%%%%%%%%%%%%%%
% 设置论文的中文摘要

% 设置中文摘要页面的论文标题及副标题的第一行。
% 此属性可选,其默认值为使用|\title|命令所设置的论文标题
 \abstracttitlea{学科}
% 设置中文摘要页面的论文标题及副标题的第二行。
% 此属性可选,其默认值为空白
 \abstracttitleb{理论研究}

%%%%%%%%%%%%%%%%%%%%%%%%%%%%%%%%%%%%%%%%%%%%%%%%%%%%%%%%%%%%%%%%%%%%%%%%%%%%%%%
% 设置论文的英文摘要

% 设置英文摘要页面的论文标题及副标题的第一行。
% 此属性可选,其默认值为使用|\englishtitle|命令所设置的论文标题
\englishabstracttitlea{Field}
% 设置英文摘要页面的论文标题及副标题的第二行。
% 此属性可选,其默认值为空白
\englishabstracttitleb{Theory Research}


%%%%%%%%%%%%%%%%%%%%%%%%%%%%%%%%%%%%%%%%%%%%%%%%%%%%%%%%%%%%%%%%%%%%%%%%%%%%%%%
\begin{document}

%%%%%%%%%%%%%%%%%%%%%%%%%%%%%%%%%%%%%%%%%%%%%%%%%%%%%%%%%%%%%%%%%%%%%%%%%%%%%%%

% 制作国家图书馆封面(博士学位论文才需要)
\makenlctitle
% 制作中文封面
\maketitle
% 制作英文封面
\makeenglishtitle


%%%%%%%%%%%%%%%%%%%%%%%%%%%%%%%%%%%%%%%%%%%%%%%%%%%%%%%%%%%%%%%%%%%%%%%%%%%%%%%
% 开始前言部分
\frontmatter

%%%%%%%%%%%%%%%%%%%%%%%%%%%%%%%%%%%%%%%%%%%%%%%%%%%%%%%%%%%%%%%%%%%%%%%%%%%%%%%
% 论文的中文摘要
\begin{abstract}
复杂网络的研究可上溯到20世纪60年代对ER网络的研究。90年后代随着Internet
的发展,以及对人类社会、通信网络、生物网络、社交网络等各领域研究的深入,
发现了小世界网络和无尺度现象等普适现象与方法。对复杂网络的定性定量的科
学理解和分析,已成为如今网络时代科学研究的一个重点课题。

在此背景下,由于云计算时代的到来,本文针对面向云计算的数据中心网络基础
设施设计中的若干问题,进行了几方面的研究。………………
% 中文关键词。关键词之间用中文全角分号隔开,末尾无标点符号。
\keywords{小世界理论;网络模型;数据中心}
\end{abstract}

%%%%%%%%%%%%%%%%%%%%%%%%%%%%%%%%%%%%%%%%%%%%%%%%%%%%%%%%%%%%%%%%%%%%%%%%%%%%%%%
% 论文的英文摘要
\begin{englishabstract}
\blindtext
% 英文关键词。关键词之间用英文半角逗号隔开,末尾无符号。
\englishkeywords{Small World; Network Model; Data Center}
\end{englishabstract}

%%%%%%%%%%%%%%%%%%%%%%%%%%%%%%%%%%%%%%%%%%%%%%%%%%%%%%%%%%%%%%%%%%%%%%%%%%%%%%%
% 生成论文目次
\cleardoublepage% ensure that the hypertarget is on the same page as the TOC heading
\hypertarget{toc}{}% set the hypertarget
\bookmark[dest=toc,level=chapter]{\contentsname}% add the bookmark
\tableofcontents

%%%%%%%%%%%%%%%%%%%%%%%%%%%%%%%%%%%%%%%%%%%%%%%%%%%%%%%%%%%%%%%%%%%%%%%%%%%%%%%
% 生成插图清单。如无需插图清单则可注释掉下述语句。
\listoffigures

%%%%%%%%%%%%%%%%%%%%%%%%%%%%%%%%%%%%%%%%%%%%%%%%%%%%%%%%%%%%%%%%%%%%%%%%%%%%%%%
% 生成附表清单。如无需附表清单则可注释掉下述语句。
\listoftables

%%%%%%%%%%%%%%%%%%%%%%%%%%%%%%%%%%%%%%%%%%%%%%%%%%%%%%%%%%%%%%%%%%%%%%%%%%%%%%%
% 开始正文部分
\mainmatter

%%%%%%%%%%%%%%%%%%%%%%%%%%%%%%%%%%%%%%%%%%%%%%%%%%%%%%%%%%%%%%%%%%%%%%%%%%%%%%%
% 学位论文的正文应以《绪论》作为第一章
\include{chapters/chapter_introduction}
\include{chapters/chapter1}
%\include{chapters/chapter2}
%\include{chapters/chapter3}
%\include{chapters/chapter4}
%\include{chapters/chapter5}
%\include{chapters/chapter6}
\include{chapters/conclusion}

%%%%%%%%%%%%%%%%%%%%%%%%%%%%%%%%%%%%%%%%%%%%%%%%%%%%%%%%%%%%%%%%%%%%%%%%%%%%%%%
\cleardoublepage
\phantomsection
\addcontentsline{toc}{chapter}{参考文献}
\nocite{*}
\printbibheading
\defbibfilter{books}{type=book or type=incollection}
\printbibliography[check=chinese,filter=books,heading=subbibliography,title={中文书籍},resetnumbers=true]
\printbibliography[check=chinese,type=article,heading=subbibliography,title={中文论文},resetnumbers=true]
\printbibliography[check=chinese,type=online,heading=subbibliography,title={中文网络资源},resetnumbers=true]
\printbibliography[check=chinese,nottype=book,nottype=incollection,nottype=article,nottype=online,heading=subbibliography,title={中文其他资源},resetnumbers=true]
\printbibliography[check=english,filter=books,heading=subbibliography,title={英文书籍},resetnumbers=true]
\printbibliography[check=english,type=article,heading=subbibliography,title={英文论文},resetnumbers=true]
\printbibliography[check=english,type=online,heading=subbibliography,title={英文网络资源},resetnumbers=true]
\printbibliography[check=english,nottype=book,nottype=incollection,nottype=article,nottype=online,heading=subbibliography,title={英文其他资源},resetnumbers=true]

%档案使用"manual",Zotero中为“报告”,导出的bib修改institution为organization

%%%%%%%%%%%%%%%%%%%%%%%%%%%%%%%%%%%%%%%%%%%%%%%%%%%%%%%%%%%%%%%%%%%%%%%%%%%%%%%
% 致谢,应放在《结论》之后
\begin{acknowledgement}
  首先感谢我的母亲韦春花对我的支持。其次感谢我的导师陈近南对我的精心指导和热心帮助。接下来,
  感谢我的师兄茅十八和风际中,他们阅读了我的论文草稿并提出了很有价值的修改建议。

  最后,感谢我亲爱的老婆们:双儿、苏荃、阿珂、沐剑屏、曾柔、建宁公主、方怡,感谢
  你们在生活上对我无微不至的关怀和照顾。我爱你们!
\end{acknowledgement}

%%%%%%%%%%%%%%%%%%%%%%%%%%%%%%%%%%%%%%%%%%%%%%%%%%%%%%%%%%%%%%%%%%%%%%%%%%%%%%%
% 书籍附件
\backmatter
%%%%%%%%%%%%%%%%%%%%%%%%%%%%%%%%%%%%%%%%%%%%%%%%%%%%%%%%%%%%%%%%%%%%%%%%%%%%%%%
% 作者简历与科研成果页,应放在backmatter之后
\begin{resume}
  \begin{firstauthor} 
  \item 《XXX》,载《XXX》(CSSCI),2021年第1期。
  \end{firstauthor}
  \begin{secondauthor}
  \item 《XXXX》,载《XXXX》, 2021年第1期。
  \end{secondauthor}
  \end{resume}

%%%%%%%%%%%%%%%%%%%%%%%%%%%%%%%%%%%%%%%%%%%%%%%%%%%%%%%%%%%%%%%%%%%%%%%%%%%%%%%
% 生成《学位论文出版授权书》页面,应放在最后一页
\makelicense

%%%%%%%%%%%%%%%%%%%%%%%%%%%%%%%%%%%%%%%%%%%%%%%%%%%%%%%%%%%%%%%%%%%%%%%%%%%%%%%
\end{document}
